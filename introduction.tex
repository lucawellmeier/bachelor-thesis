\chapter*{Introduction}
\addcontentsline{toc}{chapter}{Introduction}

Analytic combinatorics can be described as the process of encoding discrete structures as \textit{generating functions}, i.e. formal series in formal variables (formal power series, for instance), to exploit their analytic properties and thereby quantify interesting aspects of the underlying structure.
This paper is meant to be a quick introduction into the theory as presented in the book \cite{analyticcombinatorics} by Philippe Flajolet and Robert Sedgewick, which was, in fact, its first extensive treatment.

In the first part we develop tools to actually find generating functions.
Flajolet and Sedgewick summarize these as the "symbolic method":
We describe the discrete structure in question in terms of elementary concepts like sets, multisets or sequences, and then use a \textit{dictionary} to translate the description into generating functions (which could easily be carried out by a computer!).
They answer the question "How many objects are in my structure of this given size?" simply by computing series coefficients.
Moreover, we use notions from probability theory to compute more specific generating functions enabling us to solve problems like "All objects in this structure are built in some way from objects of a simpler structure. How many simpler objects can I expect this one to consist of?".

We introduce three types of generating function: ordinary, exponential and Dirichlet.
Applying all of them to the same problem generally results in different levels of "usefulness", which is intimately connected to the nature of the underlying structure.
Therefore, we will categorize combinatorial structures into rough categories (additive/multiplicative, unlabelled/labelled) to find which type to use.
Readers might also be interested in Wilf's book \cite{generatingfunctions} for a more intuitive and elementary introduction to generating functons.

The second part of this paper is concerned with analytic properties of generating functions. 
The general idea is to extract asymptotic growth properties of the quantitative properties of the structures encoded in generating function using methods from complex analysis.
These properties are absolutely necessary in the average-case analysis of algorithms, for example.

For ordinary and exponential generating functions we introduce the theory of singularity analysis, also significantly influenced by Flajolet's prior work.
It allows to extract asymptotic information of a large class of singular functions by analyzing the nature and position of their singularities.
Note, however, that there are many more tools available (the most important ones are treated in the book). 

We analyze Dirichlet generating functions using the Wiener-Ikehara Tauberian theorem.
The theorem also establishes a connection between nature and position of the singularity and quantitative properties of the structure.








% deleted foreword

% bachelor thesis
% meant to be a guide for a quick introduction to analytic combinatorics by example
% Mainly based on Flajolet
% tries to integrate dirichlet generating functions into the theory
% 5 examples, selction of covered topics is taylored to them
% -> introduced in part I and "solved" in part II



% structure
%A handful of examples are pursued throughout the text





% describe both parts of this work and the transition



%%% MUSS NOCH ABGEAENDERT WERDEN DAMIT ES INS INTRO PASST
%%% VLLT DIESER TEIL NACH DEM ERSTEN KOMMENTAR DORT UNTEN
%In practice, if we are given a counting problem, the formal methods and %constructions discussed in the discrete part will often result in good %formulae for the generating functions, even when direct attempts to solve %the problem fail. In these cases, making the transition to the continuous %can help.
%%% HIERNACH STIRLING




% introductory example of a simple generating function that can be analyzed by elementary methods

% maybe approximate the number of permutations of n objects analytically using Stirling




%%%%
% listing of all the examples by topic.